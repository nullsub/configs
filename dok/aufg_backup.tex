\chapter{Backup}
Es wurde sich für rsnapshot entschieden. rsnapshot führt shapshots mit rsync und hardlinks durch.
Da sich die meisten Daten der nodes selten ändern bleibt der Speicherplatzaufwand und Zeitaufwand für die Snapshots gering.
rsnapshot installieren
\begin{lstlisting}[style=Bash]
# sudo apt-get install rsnapshot
\end{lstlisting}
NFS mounten
\begin{lstlisting}[style=Bash]
141.76.90.115:/nfs/lctp1 /backup nfs rw 0 0 
\end{lstlisting}
In den iptables wurde der Zugriff auf die ip des nfs Nervers komplett erlaubt.
/etc/rsnapshot.conf
\begin{lstlisting}[style=Bash]
exclude /proc/
exclude /sys/
...
backup / headnode/
\end{lstlisting}

/shared und /fastfs wird von den Computenodes nicht gesichert, da diese vom headnode bereits gesichert werden.
Mit einem cronjob wird das tägliche Backup gestartet. Der headnode startet das backup um 00:00. \\
Die computenodes start ihr backup jeweils um N:00. Node1 startet somit das Backup um 01:00 Uhr.
\begin{lstlisting}[style=Bash]
0 0 * * * /usr/bin/rsnapshot daily
\end{lstlisting}
