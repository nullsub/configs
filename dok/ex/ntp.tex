\chapter{NTP}
Um alle Uhren auf den Nodes zu synchronisieren, benutzt man das NTP
(Network Time Protokol).
Der Headnode synchronisiert sich dabei mit dem NTP-Server der TU Dresden.
Computenodes synchronisieren sich mit dem Headnode. Es gilt darauf zu achten,
dass der NTP-Server auf dem Headnode nur im lokalen Netz erreichbar ist.

NTP installieren:
\begin{lstlisting}[style=Bash]
# apt-get install ntp
\end{lstlisting}
Konfiguration des NTP-Servers auf dem Headnode in /ect/ntp.conf:
\begin{lstlisting}[style=Bash]
...
server time.zih.tu-dresden.de iburst
# Zugriff vom localhost gestatten (ntpq -p)
restrict 127.0.0.1
 
# Zugriff aus dem internen Netz gestatten
restrict 192.168.2.0 mask 255.255.255.0 nomodify nopeer
...
\end{lstlisting}
\begin{lstlisting}[style=Bash]
# service ntp restart
\end{lstlisting}
Konfiguration der Clients auf den Computenodes in /etc/ntp.conf:
\begin{lstlisting}[style=Bash]
...
# Zugriff durch NTP-Server gestatten
restrict 192.168.2.1
 
# Zugriff vom localhost gestatten (ntpq -p)
restrict 127.0.0.1
...
\end{lstlisting}
\begin{lstlisting}[style=Bash]
# service ntp restart
\end{lstlisting}
