\chapter{LDAP}
Als Grundlage dient http://wiki.ubuntuusers.de/OpenLDAP.\\
Ldap, Utils und Migrationstools installieren:
\begin{lstlisting}[style=Bash]
# apt-get install slapd smbldap-tools ldap-utils
\end{lstlisting}
ldap konfigurieren:
\begin{lstlisting}[style=Bash]
# dpkg-reconfigure -plow slapd
\end{lstlisting}
\begin{tabular}{ l | l }
 option & value\\
 \hline
 domain name & cluster.local\\
 name of organization & cluster\\
 password & secret\\
 database backend to use & HDB\\
 purge remove & no\\
 Allow LDAPv2 protocol & yes\\
\end{tabular}
\section{slapd.conf}
slapd über slapd.conf konfigurieren:
\begin{lstlisting}[style=Bash]
.....
.....
suffix        "dc=cluster,dc=local"

rootdn "cn=admin,dc=cluster,dc=local"
rootpw {SSHA}52g5BxB2hCwfLR846bo+8FaBsmVoSxaA

# SASL DIGEST-MD5 authorisation replacement directive
# This parameter is in the format of:
# uid=<username>,cn=<realm>,cn=<mech>,cn=auth
# The username is taken from sasl and inserted into the ldap search 
# string in the place of $1. Your realm is supposed to be your FQDN 
# (fully qualified domain name)
authz-regexp uid=admin,cn=[^,]*,cn=auth cn=admin,dc=cluster,dc=local
authz-regexp uid=([^,]*),cn=[^,]*,cn=auth uid=$1,ou=Users,\
	dc=cluster,dc=local
authz-policy      to
# enable LDAP to help SASL to use passworts stored in LDAP database
password-hash   {CLEARTEXT}
\end{lstlisting}
Admin User hinzufügen, base.ldif:
\begin{lstlisting}[style=Bash]
objectClass: dcObject
objectClass: organization
o: cluster
dc: cluster

dn:cn=admin,dc=cluster,dc=local
objectClass: organizationalRole
cn: admin
\end{lstlisting}
\begin{lstlisting}[style=Bash]
$ sudo ldapadd -x -h localhost -W -D \
	cn=admin,dc=cluster,dc=local -f base.ldif 
\end{lstlisting}
Linux Benutzeraccounts nach LDAP migrieren.\\
init.ldif:
\begin{lstlisting}[style=Bash]
dn: ou=Users,dc=cluster,dc=local
objectClass: organizationalUnit
ou: Users

dn: ou=Groups,dc=cluster,dc=local
objectClass: organizationalUnit
ou: Groups

dn: ou=Computers,dc=cluster,dc=local
objectClass: organizationalUnit
ou: Computers

dn: ou=Idmap,dc=cluster,dc=local
objectClass: organizationalUnit
ou: Idmap
\end{lstlisting}
Hinzufügen:
\begin{lstlisting}[style=Bash]
$ sudo ldapadd -x -h localhost -W -D \
	cn=admin,dc=cluster,dc=local -f init.ldif
\end{lstlisting}
Alle Nutzer migrieren:
\begin{lstlisting}[style=Bash]
$ ./smbldap-migrate-unix-accounts -P /etc/passwd -S /etc/shadow -v
\end{lstlisting}
Alle Gruppen migrieren:
\begin{lstlisting}[style=Bash]
$ ./smbldap-migrate-unix-groups -G /etc/group -v 
\end{lstlisting}
\section{SASL}
Die Passwörter müssen als Klartext in der LDAP DB gespeichert werden damit SASL funktioniert. Darum muss für jeden Nutzer noch das Passwort gesetzt werden:
\begin{lstlisting}[style=Bash]
$ ldappasswd -x -D cn=admin,dc=cluster,dc=local -W \
	-s userpwd1 uid=user1,ou=Users,dc=cluster,dc=local 
\end{lstlisting}
Nun sollte SASL funktionieren. Dies kann man mit folgendem Befehl überprüfen:
\begin{lstlisting}[style=Bash]
$ ldapsearch -b dc=cluster,dc=local
\end{lstlisting}
Das Admin Passwort muss auch noch als Klartext hinterlegt werden.\\
Nun wird TLS eingerichtet.\\
Zertifikate erstellen; /var/ssl/ca.cfg:
\begin{lstlisting}[style=Bash]
cn = cluster
ca
cert\_signing\_key
\end{lstlisting}
\begin{lstlisting}[style=Bash]
# sh -c "certtool --generate-privkey > cakey.pem" 
# certtool --generate-self-signed --load-privkey cakey.pem \
	--template ca.cfg --outfile cacert.pem 
# sh -c "certtool --generate-privkey > ldap\_slapd\_key.pem"
\end{lstlisting}
/var/ssl/slapd.cfg:
\begin{lstlisting}[style=Bash]
organization = cluster
cn = cluster.local
tls\_www\_server
encryption\_key
signing\_key
\end{lstlisting}
Das ldap Zertifikat installieren:
\begin{lstlisting}[style=Bash]
organization = cluster
# install -D -o openldap -g openldap -m 600 
  /var/ssl/ldap\_slapd\_key.pem  /etc/ssl/private/ldap\_slapd\_key.pem
# install -D -o openldap -g openldap -m 600 
  /var/ssl/ldap\_slapd\_cert.pem  /etc/ssl/certs/ldap\_slapd\_cert.pem
# install -D -o openldap -g openldap -m 
    600 /var/ssl/cacert.pem  /etc/ssl/certs/ldap\_slapd\_cacert.pem 
\end{lstlisting}
Nun wieder /etc/ldap/slapd.conf anpassen:
\begin{lstlisting}[style=Bash]
TLSCertificateFile    /etc/ssl/certs/ldap\_slapd\_cert.pem
TLSCertificateKeyFile /etc/ssl/private/ldap\_slapd\_key.pem
TLSCACertificateFile  /etc/ssl/certs/ldap\_slapd\_cacert.pem
\end{lstlisting}
Damit slapd nun weiterhin startet muss sichergestellt sein dass der User openldap Zugriffsrechte auf /etc/ssl/* hat.\\
Der ldap Client wird nun konfiguriert um ssl zu verwenden:
\begin{lstlisting}[style=Bash]
uri ldaps://cluster.local/:636
ldap\_version 3
ssl start\_tls
tls\_cacert /etc/ssl/certs/ldap\_slapd\_cacert.pem
TLS\_REQCERT allow
\end{lstlisting}
LDAP Authentication.\\
Pakete installieren:
\begin{lstlisting}[style=Bash]
# apt-get install libnss-ldapd libpam-ldapd nslcd nscd
\end{lstlisting}
Die ldap Zugangsdaten werden in /etc/pam\_ldap.conf und in /etc/nslcd.conf geschrieben:
\begin{lstlisting}[style=Bash]
$ sudo cat /etc/nslcd.conf 
# /etc/nslcd.conf
# nslcd configuration file. See nslcd.conf(5)
# for details.

# The user and group nslcd should run as.
uid nslcd
gid nslcd

# The location at which the LDAP server(s) should be reachable.
uri     ldaps://cluster.local/:636

# The search base that will be used for all queries.
base dc=cluster,dc=local

# The LDAP protocol version to use.
ldap\_version 3

# The DN to bind with for normal lookups.
binddn cn=admin,dc=cluster,dc=local
bindpw secret
ssl on
tls\_reqcert allow
tls\_cacertfile /etc/ssl/certs/ldap\_slapd\_cacert.pem
\end{lstlisting}
Das LDAP-PAM Modul kann nun aktiviert werden: 
\begin{lstlisting}[style=Bash]
sudo pam-auth-update
\end{lstlisting}
Den root User kann man aus dem ldap Verzeichnis entfernen:
\begin{lstlisting}[style=Bash]
ldapdelete -D "cn=admin,dc=cluster,dc=local" \
	uid=root,ou=Users,dc=cluster,dc=local -W
\end{lstlisting}
\section{Client}
LDAP auf dem Computenode einrichten:
\begin{lstlisting}[style=Bash]
# apt-get install ldap-tools libpam-ldap
\end{lstlisting}
{ldap.conf pam\_ldap.conf pam\_ldap.secret nsswitch} muss analog zum Server editiert werden.\\
Damit Nutzer, welche nur im ldap Verzeichnis sind und nicht in /etc/passwd, sich auch einloggen können muss use\_authtok von /etc/pam.d/common-passwd entfernt werden.\\
Nachdem die ldap Authentication ausreichend getestet worden ist, werden die unix Accounts für alle normalen User entfernt: 
\begin{lstlisting}[style=Bash]
# userdel userN
\end{lstlisting}
\section{Anlegen neuer Benutzer}
Beim login über shh in den Headnode wird das Script makessh.sh aufgerufen.
Dort werden die keys und das home directory erzeugt:
\begin{lstlisting}[style=Bash]
@headnode:# vim /etc/pam.d/sshd
§\dotfill§
session optional pam\_exec.so seteuid 
    log=/tmp/makessh.log /usr/local/bin/makessh.sh
\end{lstlisting}
Script zum anelgen neuer Benutzer:
\lstinputlisting[style=BASH, basicstyle=\small, breaklines=true]{../aufgabe_ldap/addusers.sh}
\begin{lstlisting}[style=Bash]
# ./addusers.sh userliste.txt 
\end{lstlisting}
Legt Datei ldapusers\_new und un\_pw an.
In un\_pw sind Username und Passwort gespeichert. Dies kann nur von root gelesen werden. In ldapusers\_new sind die (falls veränderten) Usernamen gespeichert.\\
Script zum entfernen der Benutzer:
\lstinputlisting[style=BASH]{../aufgabe_ldap/delusers.sh}
Usage:
\begin{lstlisting}[style=Bash]
# ./delusers.sh ldapusers\_new
\end{lstlisting}
