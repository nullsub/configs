\chapter{Aufgabe 3 - DHCP \& DNS}
\section{DHCP}
Mit dem dhcp Server wird automatisch den Nodes ein Hostname und eine IP Adresse zugewiesen. Bei einem Nachträglichen hinzufügen von Nodes muss so nur
die Konfiguration des DHCP Servers geändert werden. Für alle Nodes kann somit das gleiche Filesystem Image genutzt werden.\\
Die Konfiguration ist in /etc/dhcp/dhcp.conf:
\begin{lstlisting}[style=Bash]
subnet 192.168.2.0 netmask 255.255.255.0 {
	range 192.168.2.240 192.168.2.250;
	option routers 192.168.2.1;
	option domain-name-servers 192.168.2.1;
}

host node0 {
	hardware ethernet 74:d4:35:84:06:f1;
	fixed-address 192.168.2.10;
	option host-name "node0";
}

host node1 {
	hardware ethernet 74:d4:35:1d:f2:91;
	fixed-address 192.168.2.11;
	option host-name "node1";
}
\end{lstlisting}
In /etc/default/isc-dhcp-server muss eth1 eingetragen werden, damit der DHCP Server nur auf eth1 läuft.\\
DHCP Server neu starten:
\begin{lstlisting}[style=Bash]
# /etc/init.d/isc-dhcp-server restart
\end{lstlisting}
In den Nodes wird /etc/network/interfaces angepasst:
\begin{lstlisting}[style=Bash]
auto eth0
iface eth0 inet dhcp
\end{lstlisting}
Damit der Headnode den lokalen DNS Server nutzt, muss /etc/dhclient.conf angepasst werden:
\begin{lstlisting}[style=Bash]
prepend domain-name-servers 127.0.0.1;
\end{lstlisting}
Da der Hostname der Nodes nicht automatisch gesetzt wird, erledigt dies ein Script in /etc/dhcp/dhclient-exit-hooks.d/.
\section{DNS}
Der DNS Server wird zum Auflösen von lokalen Domains genutzt. bind9 ist der am meistverbreitete DNS Server im Internet, daher wird auch bind9 verwendet.\\
Alle Rechner sollen über einen Fully Qualified Domain Name erreichbar sein:
\begin{lstlisting}[style=Bash]
Domains: cluster.local headnode.cluster.local nodeN.cluster.local
\end{lstlisting}
In folgende Dateien müssen die IPs und Domainnamen geschrieben werden:
\begin {itemize}
\item /etc/bind/db.cluster.local
\item /etc/bind/db.192
\item /etc/bind/named.conf.local
\item /etc/bind/named.conf.options
\end{itemize}
