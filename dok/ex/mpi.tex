\chapter{OpenMPI}
Message Passing Interface (MPI) ist ein Framework für den Datenaustausch von Prozessen. Es wird für die Parallelisierung von Anwendungen benötigt.\\
OpenMPI installieren:
\begin{lstlisting}[style=Bash]
$ sudo apt-get install openmpi-bin libopenmpi-dbg libopenmpi-dev
\end{lstlisting}
Version 1.8.4 wird nach /shared/openmpi/1.8.4 installiert:
\begin{lstlisting}[style=Bash]
$ wget open-mpi.org/software/ompi/v1.8/downloads/openmpi-1.8.4.tar.bz2
$ tar -xf openmpi-1.8.4.tar.bz2
$ cd openmpi-1.8.4 && ./configure --prefix=/shared/tools/openmpi/1.8.4
$ make -j2 && make install
\end{lstlisting}
OpenMPI wird in der Regel als Module bereitgestellt, da Anwendungen bestimmte Versionen von OpenMPI benötigen.\\
Version 1.8.4 kann als Module geladen werden:
\begin{lstlisting}[style=Bash]
$ module load openmpi/1.8.4
\end{lstlisting}
Minimalbeispiel nach /shared installieren:
\begin{lstlisting}[style=Bash]
$ git clone https://github.com/freme/MatMulExamples.git
$ make
\end{lstlisting}
In /shared/mpi\_hosts werden alle MPI Hosts abgelegt welche über ssh erreichbar sind.\\
Die Anzahl der slots gibt die Anzahl der Kerne an:
\begin{lstlisting}[style=Bash]
headnode slots=2
node0 slots=2
node1 slots=2
\end{lstlisting}
Nun kann man auf dem headnode das Beispiel starten:
\begin{lstlisting}[style=Bash]
$ cd /shared/MatMulExamples/MatMulMPICannon/
$ make runssh
\end{lstlisting}
Zur Ausführung von OpenMPI Programmen wird mpirun mit der Anzahl der parallelen Prozessen und dem Programm als Argument gestartet.\\
Anschaulich ist hierbei das helloworld Beispiel:
\begin{lstlisting}[style=Bash]
$ cd /shared/hello_mpi/
$ mpicc hello.c -o hello 
$ mpirun -np 6 --hostfile /shared/mpi_hosts hello
\end{lstlisting}
