\chapter{Batchsystem}
Mit einem Batchsystem wird dem User die Möglichkeit gegeben Programme(Jobs)
ausführen zu lassen, ohne dass eine weitere Interaktion mit dem Ihm erforderlich
ist.
Es gibt dann vielfältige Möglichkeiten wann und wie diese Jobs 
verarbeitet werden. Dabei spielen unteranderem Faktoren wie Systemauslastung und
Prioritäten eine Rolle.
Wir haben uns für Slurm entschieden, da es eine weit verbreites,
einfach zu handhabendes Batchsystem ist.\\
Slurm installieren:
\begin{lstlisting}[style=Bash]
# apt-get install slurm-llnl
$ /usr/sbin/create-munge-key
# /etc/init.d/slurm-llnl start
# /etc/init.d/munge start
\end{lstlisting}
Auf den Nodes den munge key /etc/munge/mung.key und die /etc/slurm-lnll/slurm.conf kopieren.\\
Nun wird munge gestartet.
Partitions erstellen:
\begin{lstlisting}[style=Bash]
# scontrol create PartitionName=express MaxTime=1:0:0 \
	Nodes=headnode,node0,node1
# scontrol create PartitionName=small MaxTime=1:0:0 \
	Nodes=headnode,node0,node1
\end{lstlisting}

