\chapter{Compiler}
Zusätzlich zu den bereitgestellten Compiler-Paketen werden nun weitere Compiler installiert. Die verschiedenen Versionen können dann miteinander verglichen werden.\\
Da es keine fertigen Pakete für die gesuchten Releases gibt, müssen diese von Hand compiliert werden.\\ 
Compiler werden in /shared kompiliert und nach /shared/tools installiert.\\
Sourcen downloaden:
\begin{lstlisting}[style=Bash]
# wget http://gcc.cybermirror.org/releases/gcc-4.8.3/gcc-4.8.3.tar.bz2
# wget http://gcc.cybermirror.org/releases/gcc-4.9.2/gcc-4.9.2.tar.bz2
# git clone https://github.com/gcc-mirror/gcc.git
\end{lstlisting}
Zur Kompilation von C/C++ Anwendungen werden einige Tools, Compiler und Bibliotheken benötigt:
\begin{lstlisting}[style=Bash]
$ apt-get install subversion libgmp-dev libmpfr-dev libmpc-dev \
	gcc-multilib g++ g++-multilib zip bison build-essential
\end{lstlisting}
Entpacken:
\begin{lstlisting}[style=Bash]
# tar -xf gcc-4.8.3.tar.bz2
# tar -xf gcc-4.9.2.tar.bz2
\end{lstlisting}
Beim configure muss der LIBRARY\_PATH der 32-Bit Bibliotheken extra angegeben werden.
Mit -j2 wird die Kompilation auf 2 Kernen parallel ausgeführt.\\
Nach folgendem Schema werden die Compiler kompiliert und installiert:
\begin{lstlisting}[style=Bash]
# mkdir build 
# ../configure --prefix=/shared/tools/gcc/4.x.x
# LIBRARY_PATH=/usr/lib32:$LIBRARY_PATH make -j2
# make install
\end{lstlisting}
llvm mit clang llvm installieren:
\begin{lstlisting}[style=Bash]
# svn checkout http://llvm.org/svn/llvm-project/llvm/trunk llvm
$ cd llvm/tools
$ svn co http://llvm.org/svn/llvm-project/cfe/trunk clang
$ cd ../projects
$ svn co http://llvm.org/svn/llvm-project/compiler-rt/trunk \
	compiler-rt
# svn checkout http://llvm.org/svn/llvm-project/llvm/trunk llvm
$ mkdir build && cd build && ../configure --prefix \
	/shared/tools/llvm/head
$ make -j2 && make install
\end{lstlisting}
