\chapter{OpenMPI}
openmpi installieren
\begin{lstlisting}[style=Bash]
# sudo apt-get install openmpi-bin libopenmpi-dbg libopenmpi-dev
\end{lstlisting}
Minimalbeispiel nach /fastfs installieren
\begin{lstlisting}[style=Bash]
# git clone https://github.com/freme/MatMulExamples.git
# make
\end{lstlisting}
In /fastfs/mpi\_hosts werden alle mpi hosts abgelegt welche über ssh erreichbar sind.
\begin{lstlisting}[style=Bash]
headnode slots=2
node0 slots=2
\end{lstlisting}
Nun kann man auf dem headnode das Beispiel starten.
\begin{lstlisting}[style=Bash]
# cd /fastfs/MatMulExamples/MatMulMPICannon/
# make runssh
\end{lstlisting}
Anschaulich ist auch das helloworld beispiel:
\begin{lstlisting}[style=Bash]
# cd /fastfs/hello_mpi/
# mpicc hello.c -o hello 
# mpirun -np 4 --hostfile /fastfs/mpi_hosts hello
\end{lstlisting}
