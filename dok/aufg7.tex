\chapter{LDAP}
ldap, utils und migrationstools installieren
\begin{lstlisting}[style=Bash]
# apt-get install slapd smbldap-tools ldap-utils
\end{lstlisting}
ldap konfigurieren
\begin{lstlisting}[style=Bash]
# dpkg-reconfigure -plow slapd
\end{lstlisting}
\begin{tabular}{ l | l }
 option & value\\
 \hline
 domain name & cluster.local\\
 name of organization & cluster\\
 password & secret\\
 database backend to use & HDB\\
 purge remove & no\\
 Allow LDAPv2 protocol & yes\\
\end{tabular}
\\

slap über slapd.conf konfigurieren:
\begin{lstlisting}[style=Bash]
.....
.....
suffix        "dc=cluster,dc=local"

rootdn "cn=admin,dc=cluster,dc=local"
rootpw {SSHA}52g5BxB2hCwfLR846bo+8FaBsmVoSxaA

# SASL DIGEST-MD5 authorisation replacement directive
# This parameter is in the format of:
# uid=<username>,cn=<realm>,cn=<mech>,cn=auth
# The username is taken from sasl and inserted into the ldap search 
# string in the place of $1. Your realm is supposed to be your FQDN 
# (fully qualified domain name)
authz-regexp uid=admin,cn=[^,]*,cn=auth cn=admin,dc=cluster,dc=local
authz-regexp uid=([^,]*),cn=[^,]*,cn=auth uid=$1,ou=Users,dc=cluster,dc=local
authz-policy      to
# enable LDAP to help SASL to use passworts stored in LDAP database
password-hash   {CLEARTEXT}
\end{lstlisting}

Admin user hinzufügen, base.ldif
\begin{lstlisting}[style=Bash]
dn:dc=cluster,dc=local
objectClass: dcObject
objectClass: organization
o: cluster
dc: cluster

dn:cn=admin,dc=cluster,dc=local
objectClass: organizationalRole
cn: admin
\end{lstlisting}
ldapadd -x -h localhost -W -D cn=admin,dc=cluster,dc=local -f base.ldif 

Linux Benutzeraccounts nach LDAP migrieren
init.ldif:
\begin{lstlisting}[style=Bash]
dn: ou=Users,dc=cluster,dc=local
objectClass: organizationalUnit
ou: Users

dn: ou=Groups,dc=cluster,dc=local
objectClass: organizationalUnit
ou: Groups

dn: ou=Computers,dc=cluster,dc=local
objectClass: organizationalUnit
ou: Computers

dn: ou=Idmap,dc=cluster,dc=local
objectClass: organizationalUnit
ou: Idmap
\end{lstlisting}
Hinzufügen
\begin{lstlisting}[style=Bash]
sudo ldapadd -x -h localhost -W -D cn=admin,dc=cluster,dc=local -f init.ldif
\end{lstlisting}
Alle Nutzer migrieren:
\begin{lstlisting}[style=Bash]
./smbldap-migrate-unix-accounts -P /etc/passwd -S /etc/shadow -v
\end{lstlisting}
Alle Gruppen migrieren:
\begin{lstlisting}[style=Bash]
./smbldap-migrate-unix-groups -G /etc/group -v 
\end{lstlisting}
Die passwörter müssen als klartext in der LDAP db gespeichert werden damit SASL funktioniert. Darum muss für jeden Nutzer noch das passwort gesetzt werden:
\begin{lstlisting}[style=Bash]
ldappasswd -x -D cn=admin,dc=cluster,dc=local -W -s userpwd1 uid=user1,ou=Users,dc=cluster,dc=local 
\end{lstlisting}

Die Passwörte werden i-wie noch nicht richtig als plaintext gespeichert. Also SASL funzt noch nicht.



Schrott
edit /etc/adduser.conf
\begin{lstlisting}[style=Bash]
# The DHOME variable specifies the directory containing users' home
# directories.
DHOME=/home/exports
\end{lstlisting}
edit /etc/ldap/ldap.conf 
\begin{lstlisting}[style=Bash]
BASE    dc=lctp1,dc=tud,dc=de
URI     ldap://localhost
\end{lstlisting}
start LDAP server:
\begin{lstlisting}[style=Bash]
/etc/init.d/slapd start
\end{lstlisting}
nss für LDAP
\begin{lstlisting}[style=Bash]
# apt-get install nscd
# apt-get install libnss-ldap
# dpkg-reconfigure -plow libnss-ldap
\end{lstlisting}
\begin{tabular}{ l | l }
 option & value\\
 \hline
 LDAP Server & ldapi:///127.0.0.1\\
 search base & dc=lctp1,dc=tud,dc=de\\
 LDAP Version & 3\\
 require login & no\\
 LDAP priveleges for root & yes\\
 configuration readable/writeable by owner only & yes\\
 LDAP accout for root & cn=admin,dc=lctp1,dc=tud,dc=de\\
 LDAP root account Password & secret\\
\end{tabular}
\\
nsswitch.ldap muss editiert werden TODO??\\
edit /etc/libnss-ldap.conf, uncomment rootbinddn:
\begin{lstlisting}[style=Bash]
# Use 'echo -n "mypassword" > /etc/libnss-ldap.secret' instead
# of an editor to create the file.
rootbinddn cn=admin,dc=lctp1,dc=tud,dc=de
\end{lstlisting}
write secet in /etc/libnss-ldap.secret
\begin{lstlisting}[style=Bash]
# chmod 600 /etc/libnss-ldap.secret 
\end{lstlisting}
edit /etc/nsswitch.conf:
\begin{lstlisting}[style=Bash]
passwd:         files ldap
group:          files ldap
shadow:         files ldap

hosts:          files dns ldap
networks:       files ldap
protocols:      db files

services:       db files
ethers:         db files
rpc:            db files

netgroup:       nis
\end{lstlisting}
\begin{lstlisting}[style=Bash]
# apt-get install libpam-ldap
\end{lstlisting}
\begin{tabular}{ l | l }
 option & value\\
 \hline
 LDAP Server host & 127.0.0.1\\
 Distinguished Base Name & dc=lctp1,dc=tud,dc=de\\
 LDAP version & 3\\
 Make local root Database admin & Ja\\
 Database requires logging in & Nein\\
 Root login account & cn=admin,dc=lctp1,dc=tud,dc=de\\
 Local crypt to use & crypt\\
\end{tabular}
\\
edit /etc/pam.d/passwd:
\begin{lstlisting}[style=Bash]
password required pam_ldap.so ignore_unknown_user md5
password optional pam_unix.so nullok obscure min=4 max=8 md5 try_first_pass

#@include common-password
\end{lstlisting}
edit /etc/pam.d/common-session:
\begin{lstlisting}[style=Bash]
session required pam_mkhomedir.so skel=/etc/skel/ umask=0022
session required pam_limits.so
session required pam_unix.so
session optional pam_ldap.so
\end{lstlisting}

\section*{Nutzer hinzufügen (funktioniert nicht)}
\begin{lstlisting}[style=Bash]
# ldapadd -H "ldap://127.0.0.1:389/" -cxD \
cn=admin,dc=lctp1,dc=tud,dc=de -f add-user.ldif -w secret
\end{lstlisting}
add-user.ldif:
\begin{lstlisting}[style=Bash]
# User primary group
dn: cn=sheldon,dc=lctp1,dc=tud,dc=de
cn: sheldon
objectClass: top
objectClass: posixGroup
gidNumber: 10000

# User account
dn: uid=sheldon,dc=lctp1,dc=tud,dc=de
cn: Sheldon Cooper
givenName: Sheldon
sn: Cooper
uid: sheldon
uidNumber: 10000
gidNumber: 10000
homeDirectory: /home/jsmith
mail: sheldon@dev.local
objectClass: top
objectClass: posixAccount
objectClass: shadowAccount
objectClass: inetOrgPerson
objectClass: organizationalPerson
objectClass: person
loginShell: /bin/bash
userPassword: {CRYPT}*
\end{lstlisting}
Fehler:
\begin{lstlisting}[style=Bash]
ldap_bind: Invalid credentials (49)
\end{lstlisting}
\section*{Manager hinzufügen (funktioniert nicht)}
\begin{lstlisting}[style=Bash]
# ldapadd -Y EXTERNAL -H "ldap://127.0.0.1:389/" -f manager.ldif
\end{lstlisting}
manager.ldif
\begin{lstlisting}[style=Bash]
dn: dc=lctp1,dc=tud,dc=de
objectClass: dcObject
objectclass: organization
o: lctp1.tud.de
dc: lctp1
description: My LDAP Root

dn: cn=manager,dc=lctp1,dc=tud,dc=de
objectClass: simpleSecurityObject
objectClass: organizationalRole
cn: manager
userPassword: {MD5}l5j6WiBy4I1u4C0cXC6rsw==
description: LDAP manager
\end{lstlisting}
Fehler:
\begin{lstlisting}[style=Bash]
SASL/EXTERNAL authentication started
ldap_sasl_interactive_bind_s: Unknown authentication method (-6)
	additional info: SASL(-4): no mechanism available: 
\end{lstlisting}
