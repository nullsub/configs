\chapter{Aufgabe 3 - DHCP \& DNS}
\section{DHCP}
restart with
sudo /etc/init.d/isc-dhcp-server restart
config ist in /etc/dhcp/dhcp.conf

in /etc/default/isc-dhcp-server muss noch eth1 eingetragen werden damit der DHCP server nur auf eth1 läuft

in den nodes ist /etc/network/interfaces angepasst worden

Damit der Headnode den lokalen dns server nutzt, muss /etc/dhclient.conf angepasst werden.

\begin{lstlisting}[style=Bash]
prepend domain-name-servers 127.0.0.1;
request subnet-mask, broadcast-address, time-offset, routers,
	domain-name, domain-search, host-name,
	dhcp6.domain-search,
	netbios-scope, interface-mtu,
	rfc3442-classless-static-routes, ntp-servers;
\end{lstlisting}
Der hostname der nodes wird noch nicht automatisch gesetzt. Dies kann mit einem script in /etc/dhcp/dhclient-exit-hooks.d/ gemacht werden.

\section{DNS}
Domans: cluster.local. headnode.cluster.local, nodeN.cluster.local
bind9 wird genutzt
geändert wurde:
/etc/bind/db.cluster.local
/etc/bind/db.192
/etc/bind/named.conf.local
/etc/bind/named.conf.options
