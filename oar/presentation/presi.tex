\documentclass[10pt,utf8]{beamer}

\usepackage[utf8]{inputenc}
\usepackage[ngerman]{babel}
\usepackage{amsmath}
\usepackage{bbm}

\usepackage{graphicx}
\usepackage{subfigure}
\usepackage{url}
\usepackage{hyperref}
\usepackage{eurosym}
\usepackage{listings}
\usepackage{tikz}

\usepackage{multirow}
\usepackage{colortbl}
\usepackage{booktabs}
\usepackage{setspace}
\usepackage{color, float}

\input{theme/theme}
\newcommand{\tabitem}{~~\llap{\textbullet}~~}

\lstset{ %
  backgroundcolor=\color{white},   % choose the background color; you must add \usepackage{color} or \usepackage{xcolor}
  basicstyle=\footnotesize,        % the size of the fonts that are used for the code
  breakatwhitespace=false,         % sets if automatic breaks should only happen at whitespace
  breaklines=true,                 % sets automatic line breaking
  captionpos=b,                    % sets the caption-position to bottom
  commentstyle=\color{mygreen},    % comment style
  deletekeywords={...},            % if you want to delete keywords from the given language
  escapeinside={\%*}{*)},          % if you want to add LaTeX within your code
  extendedchars=true,              % lets you use non-ASCII characters; for 8-bits encodings only, does not work with UTF-8
  frame=single,                    % adds a frame around the code
  keepspaces=true,                 % keeps spaces in text, useful for keeping indentation of code (possibly needs columns=flexible)
  keywordstyle=\color{blue},       % keyword style
  language=Octave,                 % the language of the code
  morekeywords={*,...},            % if you want to add more keywords to the set
  numbers=left,                    % where to put the line-numbers; possible values are (none, left, right)
  numbersep=5pt,                   % how far the line-numbers are from the code
  numberstyle=\tiny\color{mygray}, % the style that is used for the line-numbers
  rulecolor=\color{black},         % if not set, the frame-color may be changed on line-breaks within not-black text (e.g. comments (green here))
  showspaces=false,                % show spaces everywhere adding particular underscores; it overrides 'showstringspaces'
  showstringspaces=false,          % underline spaces within strings only
  showtabs=false,                  % show tabs within strings adding particular underscores
  stepnumber=2,                    % the step between two line-numbers. If it's 1, each line will be numbered
  stringstyle=\color{mymauve},     % string literal style
  tabsize=2,                       % sets default tabsize to 2 spaces
  title=\lstname                   % show the filename of files included with \lstinputlisting; also try caption instead of title
}


\title{Linux Cluster in Theorie und Praxis}
\subtitle{Das OAR Batchsystem}
\author{Sebastian Knauer}
\date{4. M\"arz 2015}
\institute[ZIH TUD]{Zentrum f\"ur Informationsdienste und Hochleistungsrechnen -- TU Dresden}
%\room{INF 1046}
\address{N\"othnitzer Stra{\ss}e 46}
\city{01189 Dresden}
%\phone{+49 0351 - 463 38783}
\email{s9284744@mail.zih.tu-dresden.de}

\setbeamercovered{transparent}
\begin{document}

\zihmaketitle

\begin{frame}
\frametitle{Inhalt}
	\tableofcontents
\end{frame}

\section{OAR}
\begin{frame}{\includegraphics[scale=0.22, keepaspectratio]{oar-logo.png}  OAR}
    \begin{itemize}
        \item{Open source Batchsystem}
        \item{Wird benutzt auf Grid 5000 (1000 Nodes, 8000 Cores)
            \begin{figure}
            \centering
	        \includegraphics[scale=1.0, keepaspectratio]{grid5000.jpg}
            \caption{Grid'5000}
            \end{figure}
            }
        \item{Basiert auf Datenbank (MySql oder PostgreSql) }
        \item{Scriptsprache Perl}
    \end{itemize}
\end{frame}

\section{Architekur / Installation}
\begin{frame}{Architektur / Installation}
    \begin{itemize}
        \item{OAR Installation vier Komponenten}
        \item{alle Debian repository: OAR 2.5.4-2}
    \end{itemize}
    Komponenten:
    \begin{itemize}
        \item{Server node}
        \item{Frontend node}
        \item{Computing node}
        \item{Optional: Visualisierungsserver}
    \end{itemize}
    Debian Pakete:
    \begin{itemize}
        \item{oar-server}
        \item{oar-user}
        \item{oar-node}
    \end{itemize}
%
%    \begin{tabular}{l|r}
%    ~ & debian package name\\ \hline
%    \tabitem Server node & oar-server \\
%    \tabitem Frontend node & oar-user \\
%    \tabitem Computing node & oar-node \\
%    \tabitem Optional: Visiualisierungsserver & ~ \\
%    \end{tabular}
\end{frame}

\begin{frame}{Installation/Administration}

    LCTP Cluster:

    \begin{figure}
    \centering
    \begin{tikzpicture}
        %headnode
        \draw [rounded corners,fill=orange!20](0,1) rectangle (2.5,4); 
        \node at (1.25,3.5)[font=\bfseries] {headnode};
        \draw (0,3) -- (2.5,3);
        \node at (1.25,2.5) {Server node};
        \node at (1.25,2) {Frontend node};
        %node0
        \draw [rounded corners,fill=gray!30](5,0) rectangle (7.5,2); 
        \node at (6.25,1.5)[font=\bfseries] {node0};
        \draw (5,1) -- (7.5,1);
        \node at (6.25,0.5) {computing node};
        %node1
        \draw [rounded corners,fill=gray!30](5,3) rectangle (7.5,5); 
        \node at (6.25,4.5)[font=\bfseries] {node1};
        \draw (5,4) -- (7.5,4);
        \node at (6.25,3.5) {computing node};
        %edges
        \draw (2.5,2.5) -- (5,1);
        \draw (2.5,2.5) -- (5,4);
    \end{tikzpicture}
    \caption{Architekturskizze von OAR auf LCTP G1 Cluster}
    \end{figure}

\end{frame}

\section{Benutzerinteraktion}
\begin{frame}{Benutzerinteraktion}

    \begin{itemize}
        \item{Userinteraktion auf dem Frontend node (oar-user)}
    \item{Ähnliche Befehle wie Slurm:
    \begin{table}
    \begin{tabular}{l|l}
          OAR command & Slurm command \\ \hline
        oarsub & sbatch  \\
        oardel  & scancel  \\
        oarstat & squeue  \\
        oarhold & scontrol suspend  \\
          oarresume & scontrol resume \\
    \end{tabular}
\end{table}
}
    \item{Beispiel job submission: \lstinline{frontend:~> oarsub -l /cpu=2,walltime=00:30:00 ./script.sh}
}
    \end{itemize}
\end{frame}
\begin{frame}{Features}
\begin{itemize}
    \item{CPUSet: Nodes sind sauber nach Beendigung der Jobs}
    \item{Besteffort Jobs: Besteffort Job wird gelöscht, wenn nicht-besteffort Job Ressourcen benötigt. \lstinline{oarsub -t besteffort}}
    \item{Kein Daemon auf den Nodes (Slurm ein Daemon pro node)}
\end{itemize}
\end{frame}

\section{Performance - Job Submission}
\begin{frame}{Performance - Job Submission - Slurm}
    \begin{figure}
    \centering
	\includegraphics[scale=0.18, keepaspectratio]{../output/pics/slurm.png}
    \caption{Submission Performance Slurm}
    \end{figure}
    \begin{itemize}
        \item{Submission time ca. 15ms bis 2000 jobs}
        \item{ab 2000 Jobs steigt submission time stetig (600 ms bei 30k jobs)}
    \end{itemize}
\end{frame}

\begin{frame}{Performance - Job Submission - OAR}
    \begin{figure}
    \centering
	\includegraphics[scale=0.20, keepaspectratio]{../output/pics/oar.png}
    \caption{Submission Performance OAR}
    \end{figure}
    \begin{itemize}
        \item{Submission time konstant bei ca. 400ms }
    \end{itemize}
\end{frame}

\begin{frame}{Performance Job Submission - OAR und Slurm}
    \begin{figure}
    \centering
	\includegraphics[scale=0.20, keepaspectratio]{../output/pics/oar_slurm.png}
    \caption{Submission Performance OAR}
    \end{figure}
    \begin{itemize}
        \item{ab ca. 20k jobs überholt Slurm OAR}
    \end{itemize}
\end{frame}

\subsection*{Textquellen}
\begin{frame}
	\begin{itemize}[]
	    \bibitem [0] {} Oar Official Website \url{https://oar.imag.fr}
	    \bibitem [1] {} Oar Universitys of Luxemburg \url{https://hpc.uni.lu/users/docs/oar.html}
	    \bibitem [2] {} Grid5000 Official Website \url{https://www.grid5000.fr}
	    \bibitem [3] {} Slurm Official Website \url{https://computing.llnl.gov/linux/slurm/}
    \end{itemize}
\end{frame}
\end{document}
