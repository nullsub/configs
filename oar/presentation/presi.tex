\documentclass[10pt,utf8]{beamer}

\usepackage[utf8]{inputenc}
\usepackage[ngerman]{babel}
\usepackage{amsmath}
\usepackage{bbm}

\usepackage{graphicx}
\usepackage{subfigure}
\usepackage{url}
\usepackage{hyperref}
\usepackage{eurosym}
\usepackage{listings}
\usepackage{tikz}

\usepackage{multirow}
\usepackage{colortbl}
\usepackage{booktabs}
\usepackage{setspace}
\usepackage{color, float}

\input{theme/theme}
\newcommand{\tabitem}{~~\llap{\textbullet}~~}

\title{Linux Cluster in Theorie und Praxis}
\subtitle{Das Oar Batchsystem}
\author{Sebastian Knauer}
\date{4. M\"arz 2015}
\institute[ZIH TUD]{Zentrum f\"ur Informationsdienste und Hochleistungsrechnen -- TU Dresden}
%\room{INF 1046}
\address{N\"othnitzer Stra{\ss}e 46}
\city{01189 Dresden}
%\phone{+49 0351 - 463 38783}
\email{s9284744@mail.zih.tu-dresden.de}

\setbeamercovered{transparent}
\begin{document}

\zihmaketitle

\begin{frame}
\frametitle{Inhalt}
	\tableofcontents
\end{frame}

\section{Oar}
\begin{frame}{\includegraphics[scale=0.22, keepaspectratio]{oar-logo.png}  OAR}
    \begin{itemize}
        \item{Open source Batchsystem}
        \item{Wird benutzt auf Grid 5000 (1000 Nodes, 8000 Cores)
	        \includegraphics[scale=1.0, keepaspectratio]{grid5000.jpg}
            }
        \item{Basiert auf Datenbank (MySql oder Postgree) }
        \item{Script Language Perl}
    \end{itemize}
\end{frame}

\section{Installation/Administration}
\begin{frame}{Installation/Administration}
    Komponenten:

    \begin{tabular}{l r}
    ~ & Oar 2.5.4-2 Debian Repository \\
    \tabitem Server node & oar-server \\
    \tabitem Frontend node & oar-user \\
    \tabitem Computing node & oar-node \\
    \tabitem Optional: Visiualisierungsserver & ~ \\
    \end{tabular}

    LCTP Cluster:

    \begin{tikzpicture}
        %headnode
        \draw [rounded corners,fill=orange!20](0,1) rectangle (2.5,4); 
        \node at (1.25,3.5)[font=\bfseries] {headnode};
        \draw (0,3) -- (2.5,3);
        \node at (1.25,2.5) {Server node};
        \node at (1.25,2) {Frontend node};
        %node0
        \draw [rounded corners,fill=gray!30](5,0) rectangle (7.5,2); 
        \node at (6.25,1.5)[font=\bfseries] {node0};
        \draw (5,1) -- (7.5,1);
        \node at (6.25,0.5) {computing node};
        %node1
        \draw [rounded corners,fill=gray!30](5,3) rectangle (7.5,5); 
        \node at (6.25,4.5)[font=\bfseries] {node1};
        \draw (5,4) -- (7.5,4);
        \node at (6.25,3.5) {computing node};
        %edges
        \draw (2.5,2.5) -- (5,1);
        \draw (2.5,2.5) -- (5,4);
    \end{tikzpicture}

\end{frame}

\section{User}
\begin{frame}{User}
    \begin{tabular}{c c}
          Slurm & Oar \\
          sbatch & oarsubmit \\
          scancel & oardelete \\
          squeue & oarstat \\
    \end{tabular}
\end{frame}

\section{Job Submission}
\begin{frame}{Job Submission - Slurm}
	\includegraphics[scale=0.22, keepaspectratio]{../output/pics/slurm.png}
\end{frame}

\begin{frame}{Job Submission - Oar}
	\includegraphics[scale=0.22, keepaspectratio]{../output/pics/oar.png}
\end{frame}

\begin{frame}{Job Submission - Oar und Slurm}
	\includegraphics[scale=0.22, keepaspectratio]{../output/pics/oar_slurm.png}
\end{frame}

\subsection*{Textquellen}
\begin{frame}
	\begin{itemize}[]
	    \bibitem [0] {} Oar Official Website \url{https://oar.imag.fr}
	    \bibitem [0] {} Oar Universitys of Luxemburg \url{https://hpc.uni.lu/users/docs/oar.html}
	    \bibitem [0] {} Grid5000 Official Website \url{https://www.grid5000.fr}
    \end{itemize}
\end{frame}
\end{document}
